\documentclass[journal,12pt,twocolumn]{IEEEtran}
\usepackage{hyperref}
\usepackage{amsmath}
\usepackage{mathtools}
\title{Assignment 4}
\author{JARPULA BHANU PRASAD - AI21BTECH11015}
\date{April 2022}
\newcommand\Mycomb[2][^n]{\prescript{#1\mkern-0.5mu}{}C_{#2}}
\begin{document}
\maketitle
\noindent \Large\underline{Download codes from}:\\
\noindent \large Download python code from - \href{https://github.com/jarpula-Bhanu/Assignment-4/blob/main/codes/probability.py}{Python}\\ Download latex code from - \href{https://github.com/jarpula-Bhanu/Assignment-4/blob/main/Assignment4.tex}{Latex}
\section{\large\underline{Problem-CBSE-9th Q)example 2}}
\large \noindent Q)Two coins are tossed simultaneously 500 times, and we get
\begin{align*}
Two \hspace{2mm} heads : 105 \hspace{2mm} times \\
One \hspace{2mm} head : 275 \hspace{2mm} times \\
No \hspace{2mm} head : 120 \hspace{2mm} times
\end{align*}
Find the probability of occurrence of each of these events.

\section{\large\underline{Solution}}
\noindent \underline{Theoretical probability}:\\
\noindent If two coins are thrown there are four outcomes \\
\{H H\}, \{T H\}, \{H T\} and \{T T\} \\
Now, probability of getting no head 
\begin{align}
&Pr(No \hspace{2mm} head) = \frac{\Mycomb[2]{0}}{4} = \frac{1}{4}
\end{align}
Probability of getting one head 
\begin{align}
&Pr(One \hspace{2mm} head) = \frac{\Mycomb[2]{1}}{4} = \frac{1}{2}
\end{align}
Probability of getting two head 
\begin{align}
&Pr(Two \hspace{2mm} head) = \frac{\Mycomb[2]{2}}{4} = \frac{1}{4}
\end{align}
\noindent \underline{Practical probability}:\\
\noindent Denote the outcome of the experiment by a random variable $X$ $\in$ \{0,1,2\}.\\ Where $X$ = 0 denotes occurrence of two heads, $X$ = 1 denotes occurrence of one head and $X$ = 2 denotes the occurrence of no head. Then,
\begin{align}
Pr(X = 0) = \frac{105}{500} = 0.21 \\
Pr(X = 1) = \frac{105}{500} = 0.55 \\
Pr(X = 2) = \frac{105}{500} = 0.24 
\end{align}\\
Observe that 
\begin{align*}
Pr(X = 0) + Pr(X = 1) + Pr(X = 2) = 1
\end{align*}
 Also $X$ = 0 ,$X$ = 1 and $X$ = 2 cover all the outcomes of a trial.
\end{document}